\chapter*{Preface}
\addcontentsline{toc}{chapter}{Preface}
This book is intended to provide a thorough introduction to the theory of general relativity. It is intended to serve as both a text for graduate students and a reference book for researchers. These two goals are somewhat contradictory, and to the extent that they are, part I of the book emphasizes the first goal. It treats the topics usually covered in introductory relativity courses: basic differential geometry, Einstein's equation, gravitational radiation, the standard cosmological models, and the Schwarzschild solution. More emphasis is placed on the second goal in part II of the book, which treats a wide variety of advanced topics. However, even here I have attempted to explain all the basic ideas at an introductory level.

If I were teaching a one term introductory course on general relativity, I would cover most of the material of part I together with much of appendices B and C. For a full year course, I then would choose several chapters from part II as the basis for the material covered in the second term. For example, chapters 8 and 9 and parts of chapter 12 could comprise a one term course on global methods. Chapter 7, supplemented by current literature, could serve as the basis for a course on methods for obtaining solutions. Chapter 10 and appendix E, supplemented by further reading, could be used for a course on the dynamics of general relativity. Chapter 12 (supplemented by background material from chapters 8, 9, and 11) and chapter 14 could comprise a course on the classical and quantum properties of black holes. It should be noted that the chapters in part II of the book are largely independent of each other and, for the most part, can be read out of sequence with the following major exceptions: prior reading of chapter 8 is essential for chapter 9, and chapter 8 together with parts of chapters 9 and 11 are essential for the first two sections of chapter 12.

One of the most difficult issues which arises during the writing of a book on general relativity is where in the book to present the rather substantial amount of mathematical material that is needed. Much of this material (e.g., tensor calculus and curvature) is required even for the formulation of general relativity. Some material (e.g., Lie derivatives and Killing fields) could be avoided initially but soon becomes necessary to make the discussion clearer and to simplify computations. Finally, some of the mathematical material (e.g., many of the theorems on topological spaces) is not really needed until part II of the book. If all this material were presented at the beginning of the book, it would comprise a truly formidable obstacle to learning general relativity. On the other hand, if the mathematical results were introduced only ``as needed'' in the later chapters, the mathematical discussion would become greatly fragmented and these fragments would interrupt the discussion of physical issues. The best solution I could find to this problem was to put all the mathematical material essential for the formulation of general relativity into chapters 2 and 3, and then to put the remaining mathematical topics into appendices A, B, and C. In this way, the reader can get to chapter 4 without unnecessary detours, but the discussion of all the mathematical topics remains intact and can be referenced as needed in the text. Thus, it should be emphasized that \emph{appendices A, B, and C are an essential part of this book}. The results derived in appendices B and C are used in many places throughout the book, and the definitions and results on topological spaces which are compiled in appendix A are referred to frequently in chapters 8 and 9.

One other somewhat unusual organizational feature of the book is that the Lagrangian and Hamiltonian formulations of general relativity also have been put in an appendix. Often, the Hamiltonian formulation is presented in conjunction with the initial value formulation of general relativity, but since the statements and proofs of the initial value results do not rely upon the Hamiltonian formulation, I found it logically clearer to discuss them independently. This left the Lagrangian and Hamiltonian formulations as topics which are unlinked to the material in the other chapters, too short to comprise a whole chapter on their own, and too important to omit. Thus, they ended up being treated in appendix E.

Problems are given at the end of each chapter in text. There is significant variation in the amount of thought and computation required to solve the problems, but there are very few trivial, ``mechanical'' exercises and none which are, in my opinion, inordinately difficult (i.e., I think I can solve them). Part of my purpose in giving some of the problems (particularly in the second half of the book) was to introduce important side points for which I did not want to make a detour in the text. Hence, even the reader who is determined not to do any exercises may still wish to read the problems.

I have benefited from numerous interactions with many colleagues while planning and writing this book. The influence of Robert Geroch should be apparent to readers familiar with his viewpoints on general relativity. Some of the arguments used in chapter 3 are adopted directly from the notes from a course we taught jointly in 1975. I particularly wish to thank colleagues who took the time and trouble to read parts (and, in a few cases, all) of the book and send me their suggestions for improvements. These include Abhay Ashtekar, Arvind Borde, S. Chandrasekhar, David Garfinkle, John Friedman, Robert Geroch, James Hartle, James Isenberg, Bernard Kay, Karel Kuchai, Liang Can-bin, Roger Penrose, Michael Turner, and William Unruh. Additional thanks are due David Garfinkle for checking most of the equations.

I wish to thank Susan Lancaster and Roxy Boersma for typing drafts of the manuscript and Fred Flowers for typing the final product on a word processor. Support by NSF grant PHY 80 26043 to the University of Chicago during the writing of this book is gratefully acknowledged. Finally, I wish to thank my wife, Veronica, for the considerable amount of patience displayed during the three years it took me to write this book.